\documentclass[a4paper,11pt]{article}
\usepackage[utf8]{inputenc}
%\usepackage{polski} 
\usepackage[T2A]{fontenc}
\usepackage[utf8]{inputenc}
\usepackage[english,greek,russian]{babel}
\languageattribute{greek}{polutoniko} %for accents
\usepackage{gfsporson} %for a particular Greek font
\thispagestyle{empty}

\usepackage{apacite}
\usepackage{natbib}

\begin{document}


\section*{ Задача 1 }

\section*{ Постановка }
\tab В стране Гномляндии существует N месторождений полезных ископаемых. Для кажого месторождения
известны координаты (Точка отсчета находится в левом нижнем углу государства). Богатые инвесторы (M) из соседней страны хотят купить прямоугольные участки земли в определенном месте. Требуется для каждого участка посчитать количество месторождений.

\section*{ Входные данные }
Первая строка содержит N и M - количество месторождений и инвесторов. Последующие N строк содержат положительные координаты (x и y) для кажого месторождения. Далее слудуют M строк с координатами левого нижнего и правого верхнего угла для каждого участка.
\section*{ Выходные данные }
M строк, содержащие количество месторождений на каждом из участков
\section*{ Пример }

\begin{center}
\begin{tabular}{ |c|c|c|c| } 
\hline
входные данные & выходные данные \\
\hline
\multirow 3 3 & 3\\ 
0 0 & 2 \\ 
1 1 & 2 \\
2 3 & \\
0 0 4 4 & \\
0 0 2 2 & \\
1 1 3 3 & \\
\hline
\end{tabular}
\end{center}

\pagebreak

\section*{ Задача 2 }

\section*{ Постановка }
\tab В крупной компании, занимающейся производством фото техники, потребовалось определить область, содержащую людей на общих школьных фотографиях. Программист-стажер написал скрипт, который определяет левую, правую и верхнюю сторону  прямоугольника, содержащего человека. Требуется найти объединение этих прямогльников. 

\section*{ Входные данные }
Первая строка содержит N - количество людей на фотографии. Последующие N строк содержат положительные координаты (x1, x2 и y) для каждого человека.
\section*{ Выходные данные }
Координаты полученной фигуры, отсортированные по оси x.
\section*{ Пример }

\begin{center}
\begin{tabular}{ |c|c|c|c| } 
\hline
входные данные & выходные данные \\
\hline
\multirow 5 & 2 10\\ 
2 9 10 & 3 15 \\ 
3 7 15 & 7 12 \\
5 12 12 & 12 0\\
15 20 10 & 15 10\\
19 24 8 & 20 8\\
& 24 0\\
\hline
\end{tabular}
\end{center}

\pagebreak

%%%%%%%%%%%%%%%%%%%%%%%%%%%
%%%%%%%%%%%%%%%%%%%%%%%%%%%
\end{document}
